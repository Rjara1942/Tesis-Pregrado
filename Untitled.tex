% Options for packages loaded elsewhere
\PassOptionsToPackage{unicode}{hyperref}
\PassOptionsToPackage{hyphens}{url}
\documentclass[
]{article}
\usepackage{xcolor}
\usepackage[margin=1in]{geometry}
\usepackage{amsmath,amssymb}
\setcounter{secnumdepth}{5}
\usepackage{iftex}
\ifPDFTeX
  \usepackage[T1]{fontenc}
  \usepackage[utf8]{inputenc}
  \usepackage{textcomp} % provide euro and other symbols
\else % if luatex or xetex
  \usepackage{unicode-math} % this also loads fontspec
  \defaultfontfeatures{Scale=MatchLowercase}
  \defaultfontfeatures[\rmfamily]{Ligatures=TeX,Scale=1}
\fi
\usepackage{lmodern}
\ifPDFTeX\else
  % xetex/luatex font selection
\fi
% Use upquote if available, for straight quotes in verbatim environments
\IfFileExists{upquote.sty}{\usepackage{upquote}}{}
\IfFileExists{microtype.sty}{% use microtype if available
  \usepackage[]{microtype}
  \UseMicrotypeSet[protrusion]{basicmath} % disable protrusion for tt fonts
}{}
\makeatletter
\@ifundefined{KOMAClassName}{% if non-KOMA class
  \IfFileExists{parskip.sty}{%
    \usepackage{parskip}
  }{% else
    \setlength{\parindent}{0pt}
    \setlength{\parskip}{6pt plus 2pt minus 1pt}}
}{% if KOMA class
  \KOMAoptions{parskip=half}}
\makeatother
\usepackage{graphicx}
\makeatletter
\newsavebox\pandoc@box
\newcommand*\pandocbounded[1]{% scales image to fit in text height/width
  \sbox\pandoc@box{#1}%
  \Gscale@div\@tempa{\textheight}{\dimexpr\ht\pandoc@box+\dp\pandoc@box\relax}%
  \Gscale@div\@tempb{\linewidth}{\wd\pandoc@box}%
  \ifdim\@tempb\p@<\@tempa\p@\let\@tempa\@tempb\fi% select the smaller of both
  \ifdim\@tempa\p@<\p@\scalebox{\@tempa}{\usebox\pandoc@box}%
  \else\usebox{\pandoc@box}%
  \fi%
}
% Set default figure placement to htbp
\def\fps@figure{htbp}
\makeatother
\setlength{\emergencystretch}{3em} % prevent overfull lines
\providecommand{\tightlist}{%
  \setlength{\itemsep}{0pt}\setlength{\parskip}{0pt}}
\usepackage{booktabs}
\usepackage{longtable}
\usepackage{array}
\usepackage{multirow}
\usepackage{wrapfig}
\usepackage{float}
\usepackage{colortbl}
\usepackage{pdflscape}
\usepackage{tabu}
\usepackage{threeparttable}
\usepackage{threeparttablex}
\usepackage[normalem]{ulem}
\usepackage{makecell}
\usepackage{xcolor}
\usepackage{bookmark}
\IfFileExists{xurl.sty}{\usepackage{xurl}}{} % add URL line breaks if available
\urlstyle{same}
\hypersetup{
  pdftitle={Informe Preliminar de Estimación de Precios Ex-Vessel},
  pdfauthor={Tesis Pesquería Pelágica},
  hidelinks,
  pdfcreator={LaTeX via pandoc}}

\title{Informe Preliminar de Estimación de Precios Ex-Vessel}
\usepackage{etoolbox}
\makeatletter
\providecommand{\subtitle}[1]{% add subtitle to \maketitle
  \apptocmd{\@title}{\par {\large #1 \par}}{}{}
}
\makeatother
\subtitle{Evidencia de Rigidez de Oferta y Endogeneidad en la Zona
Centro-Sur}
\author{Tesis Pesquería Pelágica}
\date{2026-01-07}

\begin{document}
\maketitle

{
\setcounter{tocdepth}{2}
\tableofcontents
}
El objetivo es verificar empíricamente la existencia de rigidez de
oferta y problemas de endogeneidad que justifiquen el uso del modelo
IAIDS.

Se procesaron las bases de datos de desembarque mensuales (2011-2024)
para tres flotas (Botes, Lanchas, Industrial). Se aplicaron los
siguientes filtros para definir el Mercado Relevante: Se consideraron
exclusivamente los desembarques y precios de las regiones V, VI, VII,
VIII, IX, XIV, X y XVI y se utilizaron únicamente precios de materia
prima con destino a Harina (``ANIMAL'').

Se estimó la elasticidad precio-cantidad (ϵ) mediante Mínimos Cuadrados
Ordinarios (OLS) para cada segmento de flota.

\begin{longtable}[t]{llrrrr}
\caption{\label{tab:unnamed-chunk-2}Estimación Desagregada por Flota}\\
\toprule
Origen\_Flota & Origen\_Especie & Obs & Flexibilidad & P\_Valor & R2\\
\midrule
\cellcolor{gray!10}{Botes} & \cellcolor{gray!10}{Anchoveta} & \cellcolor{gray!10}{69} & \cellcolor{gray!10}{0.0048} & \cellcolor{gray!10}{0.8855} & \cellcolor{gray!10}{0.000}\\
Botes & Jurel & 29 & 0.0431 & 0.5904 & 0.011\\
\cellcolor{gray!10}{Botes} & \cellcolor{gray!10}{Sardina} & \cellcolor{gray!10}{98} & \cellcolor{gray!10}{-0.0130} & \cellcolor{gray!10}{0.6485} & \cellcolor{gray!10}{0.002}\\
Industrial & Anchoveta & 51 & -0.0496 & 0.4820 & 0.010\\
\cellcolor{gray!10}{Industrial} & \cellcolor{gray!10}{Jurel} & \cellcolor{gray!10}{32} & \cellcolor{gray!10}{0.0665} & \cellcolor{gray!10}{0.1136} & \cellcolor{gray!10}{0.081}\\
\addlinespace
Industrial & Sardina & 61 & -0.0777 & 0.0444 & 0.067\\
\cellcolor{gray!10}{Lanchas} & \cellcolor{gray!10}{Anchoveta} & \cellcolor{gray!10}{91} & \cellcolor{gray!10}{-0.0207} & \cellcolor{gray!10}{0.5044} & \cellcolor{gray!10}{0.005}\\
Lanchas & Jurel & 30 & 0.0850 & 0.0270 & 0.163\\
\cellcolor{gray!10}{Lanchas} & \cellcolor{gray!10}{Sardina} & \cellcolor{gray!10}{101} & \cellcolor{gray!10}{-0.0178} & \cellcolor{gray!10}{0.5439} & \cellcolor{gray!10}{0.004}\\
\bottomrule
\end{longtable}

Debido a que el mercado de harina es integrado, se realizó una
estimación agregando la oferta total

\begin{longtable}[t]{lrrrr}
\caption{\label{tab:unnamed-chunk-3}Estimación Agregada (Mercado Total)}\\
\toprule
Origen\_Especie & Obs & Flexibilidad & P\_Valor & R2\\
\midrule
\cellcolor{gray!10}{Anchoveta} & \cellcolor{gray!10}{91} & \cellcolor{gray!10}{-0.0315} & \cellcolor{gray!10}{0.3882} & \cellcolor{gray!10}{0.008}\\
Jurel & 32 & 0.0907 & 0.0801 & 0.099\\
\cellcolor{gray!10}{Sardina} & \cellcolor{gray!10}{101} & \cellcolor{gray!10}{-0.0104} & \cellcolor{gray!10}{0.7478} & \cellcolor{gray!10}{0.001}\\
\bottomrule
\end{longtable}

Discusión de Resultados

La Tabla 2 revela fallas estructurales en el modelo OLS simple que
validan la hipótesis central de la tesis:

Jurel (+r
res\_agg\(Flexibilidad[res_agg\)Origen\_Especie==``Jurel''{]}): El
coeficiente es positivo y significativo al 10\% (p=r
res\_agg\(P_Valor[res_agg\)Origen\_Especie==``Jurel''{]}). Esto viola la
ley de demanda y evidencia simultaneidad: la demanda internacional
impulsa tanto el precio como la captura, generando un sesgo positivo que
OLS no puede corregir.

Sardina (-0.01): La pérdida de significancia en el modelo agregado
sugiere un sesgo de atenuación por endogeneidad.

\end{document}
