% Options for packages loaded elsewhere
\PassOptionsToPackage{unicode}{hyperref}
\PassOptionsToPackage{hyphens}{url}
\documentclass[
]{article}
\usepackage{xcolor}
\usepackage[margin=1in]{geometry}
\usepackage{amsmath,amssymb}
\setcounter{secnumdepth}{-\maxdimen} % remove section numbering
\usepackage{iftex}
\ifPDFTeX
  \usepackage[T1]{fontenc}
  \usepackage[utf8]{inputenc}
  \usepackage{textcomp} % provide euro and other symbols
\else % if luatex or xetex
  \usepackage{unicode-math} % this also loads fontspec
  \defaultfontfeatures{Scale=MatchLowercase}
  \defaultfontfeatures[\rmfamily]{Ligatures=TeX,Scale=1}
\fi
\usepackage{lmodern}
\ifPDFTeX\else
  % xetex/luatex font selection
\fi
% Use upquote if available, for straight quotes in verbatim environments
\IfFileExists{upquote.sty}{\usepackage{upquote}}{}
\IfFileExists{microtype.sty}{% use microtype if available
  \usepackage[]{microtype}
  \UseMicrotypeSet[protrusion]{basicmath} % disable protrusion for tt fonts
}{}
\makeatletter
\@ifundefined{KOMAClassName}{% if non-KOMA class
  \IfFileExists{parskip.sty}{%
    \usepackage{parskip}
  }{% else
    \setlength{\parindent}{0pt}
    \setlength{\parskip}{6pt plus 2pt minus 1pt}}
}{% if KOMA class
  \KOMAoptions{parskip=half}}
\makeatother
\usepackage{color}
\usepackage{fancyvrb}
\newcommand{\VerbBar}{|}
\newcommand{\VERB}{\Verb[commandchars=\\\{\}]}
\DefineVerbatimEnvironment{Highlighting}{Verbatim}{commandchars=\\\{\}}
% Add ',fontsize=\small' for more characters per line
\usepackage{framed}
\definecolor{shadecolor}{RGB}{248,248,248}
\newenvironment{Shaded}{\begin{snugshade}}{\end{snugshade}}
\newcommand{\AlertTok}[1]{\textcolor[rgb]{0.94,0.16,0.16}{#1}}
\newcommand{\AnnotationTok}[1]{\textcolor[rgb]{0.56,0.35,0.01}{\textbf{\textit{#1}}}}
\newcommand{\AttributeTok}[1]{\textcolor[rgb]{0.13,0.29,0.53}{#1}}
\newcommand{\BaseNTok}[1]{\textcolor[rgb]{0.00,0.00,0.81}{#1}}
\newcommand{\BuiltInTok}[1]{#1}
\newcommand{\CharTok}[1]{\textcolor[rgb]{0.31,0.60,0.02}{#1}}
\newcommand{\CommentTok}[1]{\textcolor[rgb]{0.56,0.35,0.01}{\textit{#1}}}
\newcommand{\CommentVarTok}[1]{\textcolor[rgb]{0.56,0.35,0.01}{\textbf{\textit{#1}}}}
\newcommand{\ConstantTok}[1]{\textcolor[rgb]{0.56,0.35,0.01}{#1}}
\newcommand{\ControlFlowTok}[1]{\textcolor[rgb]{0.13,0.29,0.53}{\textbf{#1}}}
\newcommand{\DataTypeTok}[1]{\textcolor[rgb]{0.13,0.29,0.53}{#1}}
\newcommand{\DecValTok}[1]{\textcolor[rgb]{0.00,0.00,0.81}{#1}}
\newcommand{\DocumentationTok}[1]{\textcolor[rgb]{0.56,0.35,0.01}{\textbf{\textit{#1}}}}
\newcommand{\ErrorTok}[1]{\textcolor[rgb]{0.64,0.00,0.00}{\textbf{#1}}}
\newcommand{\ExtensionTok}[1]{#1}
\newcommand{\FloatTok}[1]{\textcolor[rgb]{0.00,0.00,0.81}{#1}}
\newcommand{\FunctionTok}[1]{\textcolor[rgb]{0.13,0.29,0.53}{\textbf{#1}}}
\newcommand{\ImportTok}[1]{#1}
\newcommand{\InformationTok}[1]{\textcolor[rgb]{0.56,0.35,0.01}{\textbf{\textit{#1}}}}
\newcommand{\KeywordTok}[1]{\textcolor[rgb]{0.13,0.29,0.53}{\textbf{#1}}}
\newcommand{\NormalTok}[1]{#1}
\newcommand{\OperatorTok}[1]{\textcolor[rgb]{0.81,0.36,0.00}{\textbf{#1}}}
\newcommand{\OtherTok}[1]{\textcolor[rgb]{0.56,0.35,0.01}{#1}}
\newcommand{\PreprocessorTok}[1]{\textcolor[rgb]{0.56,0.35,0.01}{\textit{#1}}}
\newcommand{\RegionMarkerTok}[1]{#1}
\newcommand{\SpecialCharTok}[1]{\textcolor[rgb]{0.81,0.36,0.00}{\textbf{#1}}}
\newcommand{\SpecialStringTok}[1]{\textcolor[rgb]{0.31,0.60,0.02}{#1}}
\newcommand{\StringTok}[1]{\textcolor[rgb]{0.31,0.60,0.02}{#1}}
\newcommand{\VariableTok}[1]{\textcolor[rgb]{0.00,0.00,0.00}{#1}}
\newcommand{\VerbatimStringTok}[1]{\textcolor[rgb]{0.31,0.60,0.02}{#1}}
\newcommand{\WarningTok}[1]{\textcolor[rgb]{0.56,0.35,0.01}{\textbf{\textit{#1}}}}
\usepackage{longtable,booktabs,array}
\usepackage{calc} % for calculating minipage widths
% Correct order of tables after \paragraph or \subparagraph
\usepackage{etoolbox}
\makeatletter
\patchcmd\longtable{\par}{\if@noskipsec\mbox{}\fi\par}{}{}
\makeatother
% Allow footnotes in longtable head/foot
\IfFileExists{footnotehyper.sty}{\usepackage{footnotehyper}}{\usepackage{footnote}}
\makesavenoteenv{longtable}
\usepackage{graphicx}
\makeatletter
\newsavebox\pandoc@box
\newcommand*\pandocbounded[1]{% scales image to fit in text height/width
  \sbox\pandoc@box{#1}%
  \Gscale@div\@tempa{\textheight}{\dimexpr\ht\pandoc@box+\dp\pandoc@box\relax}%
  \Gscale@div\@tempb{\linewidth}{\wd\pandoc@box}%
  \ifdim\@tempb\p@<\@tempa\p@\let\@tempa\@tempb\fi% select the smaller of both
  \ifdim\@tempa\p@<\p@\scalebox{\@tempa}{\usebox\pandoc@box}%
  \else\usebox{\pandoc@box}%
  \fi%
}
% Set default figure placement to htbp
\def\fps@figure{htbp}
\makeatother
\setlength{\emergencystretch}{3em} % prevent overfull lines
\providecommand{\tightlist}{%
  \setlength{\itemsep}{0pt}\setlength{\parskip}{0pt}}
\usepackage{bookmark}
\IfFileExists{xurl.sty}{\usepackage{xurl}}{} % add URL line breaks if available
\urlstyle{same}
\hypersetup{
  hidelinks,
  pdfcreator={LaTeX via pandoc}}

\author{}
\date{\vspace{-2.5em}}

\begin{document}

\section{ANÁLISIS EXHAUSTIVO: INTEGRACIÓN PRECIOS +
DESEMBARQUES}\label{anuxe1lisis-exhaustivo-integraciuxf3n-precios-desembarques}

\subsection{RESUMEN EJECUTIVO}\label{resumen-ejecutivo}

Tu script de integración (\texttt{03\_integracion\_base\_final.R}) es
\textbf{EXCELENTE} y demuestra comprensión profunda del problema.
Generaste 3 bases estratégicamente diseñadas, cada una con un propósito
específico.

\subsubsection{Bases Generadas}\label{bases-generadas}

\begin{longtable}[]{@{}
  >{\raggedright\arraybackslash}p{(\linewidth - 8\tabcolsep) * \real{0.1154}}
  >{\raggedright\arraybackslash}p{(\linewidth - 8\tabcolsep) * \real{0.2885}}
  >{\raggedright\arraybackslash}p{(\linewidth - 8\tabcolsep) * \real{0.1346}}
  >{\raggedright\arraybackslash}p{(\linewidth - 8\tabcolsep) * \real{0.1346}}
  >{\raggedright\arraybackslash}p{(\linewidth - 8\tabcolsep) * \real{0.3269}}@{}}
\toprule\noalign{}
\begin{minipage}[b]{\linewidth}\raggedright
Base
\end{minipage} & \begin{minipage}[b]{\linewidth}\raggedright
Observaciones
\end{minipage} & \begin{minipage}[b]{\linewidth}\raggedright
Nivel
\end{minipage} & \begin{minipage}[b]{\linewidth}\raggedright
Panel
\end{minipage} & \begin{minipage}[b]{\linewidth}\raggedright
Uso Recomendado
\end{minipage} \\
\midrule\noalign{}
\endhead
\bottomrule\noalign{}
\endlastfoot
\textbf{Regional} & 341 & Mes-Región-Especie & Desbalanceado &
Heterogeneidad espacial \\
\textbf{Macro Completa} & 204 & Mes-Especie & Desbalanceado & Análisis
general \\
\textbf{Macro Balanceada} & 54 & Mes-Especie & \textbf{PERFECTAMENTE
BALANCEADO} & \textbf{IAIDS} \\
\end{longtable}

\begin{center}\rule{0.5\linewidth}{0.5pt}\end{center}

\subsection{HALLAZGOS CRÍTICOS}\label{hallazgos-cruxedticos}

\subsubsection{1. PROBLEMA DE COMPLETITUD: SOLO 18 MESES
BALANCEADOS}\label{problema-de-completitud-solo-18-meses-balanceados}

\textbf{HALLAZGO MÁS IMPORTANTE:}

De 107 meses únicos con datos, \textbf{SOLO 18 meses (16.8\%)} tienen
las 3 especies simultáneamente.

\paragraph{Distribución Temporal de Meses
Balanceados}\label{distribuciuxf3n-temporal-de-meses-balanceados}

\begin{verbatim}
Año    Meses Balanceados
----   -----------------
2012   1  (Marzo)
2013   0  
2014   0  
2015   0  
2016   1  (Marzo)
2017   1  (Mayo)
2018   1  (Febrero)
2019   0  
2020   5  (Feb, Mar, Abr, May, Jun)  ← Mejor año
2021   0  
2022   2  (Nov, Dic)
2023   4  (Mar, Abr, May, Jun)
2024   3  (Mar, Abr, May)
----   -----------------
TOTAL  18 meses en 13 años
\end{verbatim}

** Implicaciones:**

\begin{enumerate}
\def\labelenumi{\arabic{enumi}.}
\tightlist
\item
  \textbf{Base balanceada es PEQUEÑA:} Solo 54 observaciones (18 meses ×
  3 especies)
\item
  \textbf{Periodo NO es continuo:} Hay brechas de años completos
  (2013-2015, 2019, 2021)
\item
  \textbf{Sesgo temporal:} 78\% de datos están en 2020-2024 (período
  reciente)
\item
  \textbf{Sesgo estacional:} Concentración en marzo-junio (temporada
  alta)
\end{enumerate}

\begin{center}\rule{0.5\linewidth}{0.5pt}\end{center}

\subsubsection{2. CALIDAD DEL MATCH
PRECIO-CANTIDAD}\label{calidad-del-match-precio-cantidad}

\paragraph{Match Regional (nivel más
desagregado)}\label{match-regional-nivel-muxe1s-desagregado}

\begin{verbatim}
Match Rate: 341 / 1,468 = 23.2%

Obs con PRECIO y CANTIDAD: 341  
Obs solo con PRECIO:       0
Obs solo con CANTIDAD:     1,127
Total:                     1,468
\end{verbatim}

\textbf{Interpretación:} - \textbf{Excelente:} No hay precios huérfanos
(todas las obs de precio tienen cantidad) - \textbf{Esperado:} Muchos
desembarques sin precio (encuestas de precio son muestrales)

\paragraph{Match Macrozonal}\label{match-macrozonal}

\begin{verbatim}
Precios macro:       204 obs
Desembarques macro:  464 obs
Match:               204 obs (100% de precios)
\end{verbatim}

\textbf{Conclusión:} El match es \textbf{perfecto por diseño}. Los 204
meses-especie con precio SIEMPRE tienen desembarques.

\begin{center}\rule{0.5\linewidth}{0.5pt}\end{center}

\subsubsection{3. CONSISTENCIA DE
AGREGACIÓN}\label{consistencia-de-agregaciuxf3n}

\paragraph{Precios: Regional →
Macrozonal}\label{precios-regional-macrozonal}

\begin{verbatim}
Diferencia media:   0.00%  
Diferencia máxima:  0.00%  
\end{verbatim}

\textbf{Perfecto:} La ponderación por \texttt{Q\_MUESTRA\_PLANTAS}
funciona correctamente.

\paragraph{Cantidades: Regional →
Macrozonal}\label{cantidades-regional-macrozonal}

\begin{verbatim}
Diferencia media:   19.39%  
Diferencia máxima:  99.88%  
\end{verbatim}

\textbf{PROBLEMA DETECTADO:}

Esta discrepancia se debe a que: - \textbf{Base regional:} Solo tiene
los 341 mes-región-especie que TAMBIÉN tienen precio - \textbf{Base
macro:} Suma TODOS los desembarques (incluso sin precio)

\textbf{Ejemplo del problema:}

\begin{verbatim}
Mes 2020-03, ANCHOVETA:
 Regional completo suma: 50,000 ton (de regiones con precio)
 Macro suma:            100,000 ton (de TODAS las regiones)

Diferencia: 50% (porque faltan regiones sin datos de precio)
\end{verbatim}

** Implicación:** - Para análisis de \textbf{mercado de precios}, usa
base regional o macro (son coherentes) - Para análisis de \textbf{oferta
total}, la macro completa captura MÁS desembarques

\begin{center}\rule{0.5\linewidth}{0.5pt}\end{center}

\subsubsection{4. DISTRIBUCIÓN DE MESES POR
COMPLETITUD}\label{distribuciuxf3n-de-meses-por-completitud}

\paragraph{Base Macrozonal Completa (204 obs en 107
meses)}\label{base-macrozonal-completa-204-obs-en-107-meses}

\begin{verbatim}
Meses con:
 1 especie:  28 meses (26.2%)
2 especies: 61 meses (57.0%)
 3 especies: 18 meses (16.8%)   Solo estos en base balanceada
\end{verbatim}

\textbf{Problema IAIDS:} - El modelo IAIDS requiere que \textbf{TODAS}
las especies estén presentes - El 83.2\% de tus datos NO califica para
IAIDS - Alternativas: 1. \textbf{Usar base balanceada} (54 obs, limitado
pero correcto) 2. \textbf{Imputar precios faltantes} (riesgoso,
introduce supuestos) 3. \textbf{Modelo AIDS parcial} (estimar con
especies disponibles en cada período)

\begin{center}\rule{0.5\linewidth}{0.5pt}\end{center}

\subsubsection{5. ANÁLISIS DE PRECIOS}\label{anuxe1lisis-de-precios}

\paragraph{Estadísticas Descriptivas (Base
Balanceada)}\label{estaduxedsticas-descriptivas-base-balanceada}

\begin{longtable}[]{@{}lllllll@{}}
\toprule\noalign{}
Especie & N & Media (\$/ton) & SD & CV & Mín & Máx \\
\midrule\noalign{}
\endhead
\bottomrule\noalign{}
\endlastfoot
\textbf{Anchoveta} & 18 & 151,781 & 36,592 & 24.1\% & 62,695 &
215,570 \\
\textbf{Jurel} & 18 & 219,157 & 76,694 & \textbf{35.0\%} & 59,000 &
300,000 \\
\textbf{Sardina} & 18 & 150,271 & 35,260 & 23.5\% & 63,621 & 215,570 \\
\end{longtable}

\textbf{Hallazgos:} 1. \textbf{Jurel es MÁS CARO:} +45\% vs
anchoveta/sardina (mayor contenido graso) 2. \textbf{Jurel es MÁS
VOLÁTIL:} CV=35\% (mayor variabilidad de mercado) 3. \textbf{Anchoveta y
sardina tienen precios similares} (sustitutos en reducción)

\paragraph{Evolución Temporal de
Precios}\label{evoluciuxf3n-temporal-de-precios}

\begin{verbatim}
Año    Anchoveta   Jurel    Sardina
----   ---------   -------  --------
2012   $62,695     $59,000  $63,621  ← Precios bajos (referencia)
2016   $120,801    $114,000 $121,408 ← Dobla (↑90%)
2017   $93,863     $120,000 $103,010 ← Caída
2018   $215,570    $230,000 $215,570 ← Pico histórico (↑245%)
2020   $149,826    $174,365 $142,255 ← Normalización
2022   $140,000    $245,000 $140,000
2023   $160,000    $290,000 $160,000 ← Jurel sube solo
2024   $190,000    $300,000 $190,000 ← Máximo jurel
\end{verbatim}

\textbf{Tendencias:} - \textbf{2012-2018:} Crecimiento explosivo
(+245\%) - \textbf{2018:} Pico sincronizado (¿boom de harina
internacional?) - \textbf{2019-2024:} Divergencia (jurel sube,
sardina/anchoveta estables)

\begin{center}\rule{0.5\linewidth}{0.5pt}\end{center}

\subsubsection{6. ANÁLISIS DE
CANTIDADES}\label{anuxe1lisis-de-cantidades}

\paragraph{Desembarques por Año (Base
Balanceada)}\label{desembarques-por-auxf1o-base-balanceada}

\begin{verbatim}
Año    Anchoveta  Jurel     Sardina
----   ---------  --------  ---------
2012   6,730      25,882    246,996   ← Boom sardina
2016   30,622     23,188    40,715
2017   8,768      70,295    30,950
2018   2,142      50,864    15,462    ← Colapso sardina
2020   118,707    359,308   191,763   ← Boom todo
2022   23,647     58,053    33,226
2023   156,126    360,130   358,354   ← Máximo histórico
2024   97,533     312,394   83,439
\end{verbatim}

\textbf{Hallazgos:} 1. \textbf{Sardina colapsa:} 247k (2012) → 15k
(2018) = -94\% 2. \textbf{Jurel explota:} 26k (2012) → 360k (2023) =
+1,290\% 3. \textbf{Anchoveta crece:} 7k (2012) → 156k (2023) = +2,130\%
4. \textbf{2020 es excepcional:} Boom simultáneo en todas las especies

\begin{center}\rule{0.5\linewidth}{0.5pt}\end{center}

\subsubsection{7. CORRELACIÓN PRECIO-CANTIDAD
⚠}\label{correlaciuxf3n-precio-cantidad}

\paragraph{Correlaciones por Especie (Base
Balanceada)}\label{correlaciones-por-especie-base-balanceada}

\begin{verbatim}
Especie          Corr(P,Q)   Corr(log P, log Q)
--------------   ---------   ------------------
ANCHOVETA        +0.357      +0.282
JUREL            +0.577      +0.295
SARDINA COMÚN    -0.419      -0.377  ← ¡NEGATIVA!
\end{verbatim}

** HALLAZGO CRÍTICO:**

Las correlaciones son \textbf{OPUESTAS a la teoría económica clásica}:

\textbf{Esperado (Curva de oferta hacia arriba):}

\begin{verbatim}
↑ Cantidad → ↓ Precio (exceso de oferta)
Correlación negativa
\end{verbatim}

\textbf{Observado:}

\begin{verbatim}
Anchoveta: ↑ Cantidad → ↑ Precio (+0.36)  
Jurel:     ↑ Cantidad → ↑ Precio (+0.58)  
Sardina:   ↑ Cantidad → ↓ Precio (-0.42)  ✓ (coherente)
\end{verbatim}

\textbf{Posibles Explicaciones:}

\begin{enumerate}
\def\labelenumi{\arabic{enumi}.}
\tightlist
\item
  \textbf{Variable omitida: Precio internacional de harina}

  \begin{itemize}
  \tightlist
  \item
    Si precio mundial ↑ → incentiva captura ↑ Y precio local ↑
  \item
    Ambos efectos van en la misma dirección
  \end{itemize}
\item
  \textbf{Demanda muy elástica}

  \begin{itemize}
  \tightlist
  \item
    Mercado de reducción puede absorber toda la oferta
  \item
    Precio está anclado al mercado internacional
  \end{itemize}
\item
  \textbf{Efectos temporales confundidos}

  \begin{itemize}
  \tightlist
  \item
    Tendencia al alza en ambas variables (spurious correlation)
  \item
    Necesitas controlar por tendencia temporal
  \end{itemize}
\item
  \textbf{Calidad del recurso}

  \begin{itemize}
  \tightlist
  \item
    Años con alto reclutamiento → + cantidad + mejor calidad → + precio
  \item
    \% grasa varía por temporada
  \end{itemize}
\end{enumerate}

\textbf{Recomendación:} - Incluir \textbf{precio internacional de
harina} como variable de control - Usar \textbf{diferencias primeras} o
\textbf{tendencia temporal} - Modelo con \textbf{efectos fijos por año}

\begin{center}\rule{0.5\linewidth}{0.5pt}\end{center}

\subsubsection{8. DISPERSIÓN REGIONAL DE
PRECIOS}\label{dispersiuxf3n-regional-de-precios}

\paragraph{Alta Dispersión (CV \textgreater{}
30\%)}\label{alta-dispersiuxf3n-cv-30}

\textbf{Frecuencia:} 8 de 204 observaciones (3.9\%)

\textbf{Top casos:}

\begin{verbatim}
Año-Mes  Especie    CV    N_Regiones  Precio_Medio
-------- ---------  ----  ----------  ------------
2019-07  Sardina    84.5% 3           $131,934
2019-11  Sardina    78.7% 3           $144,046
2019-12  Sardina    78.2% 3           $142,656
2019-07  Anchoveta  42.8% 3           $144,208
\end{verbatim}

\textbf{Patrón:} - \textbf{Concentrado en 2019} (¿evento específico?) -
\textbf{Sardina es más volátil regionalmente} - \textbf{3 regiones con
precios muy diferentes} (segmentación de mercado)

\textbf{Interpretación:} - Posible \textbf{calidad diferencial} por
región - O \textbf{mercados parcialmente segmentados} - O \textbf{error
de medición} (encuestas pequeñas)

\begin{center}\rule{0.5\linewidth}{0.5pt}\end{center}

\subsubsection{9. ANÁLISIS DE OUTLIERS}\label{anuxe1lisis-de-outliers}

\paragraph{Frecuencia General}\label{frecuencia-general}

\begin{verbatim}
Base Macro Completa (204 obs):
├─ Outliers de PRECIO:    1  (0.5%)   ✓ Muy pocos
└─ Outliers de CANTIDAD:  150 (73.5%)  ¡Mayoría!
\end{verbatim}

\textbf{💡 Hallazgo Importante:}

\begin{itemize}
\tightlist
\item
  \textbf{Precios son muy estables} (1 outlier en 204 obs)
\item
  \textbf{Cantidades son muy volátiles} (150 outliers en 204 obs)
\end{itemize}

\textbf{Implicación:} - Los ``outliers'' de cantidad son \textbf{EVENTOS
REALES} (pulsos de captura, temporada excepcional) - \textbf{NO
eliminarlos} (son informativos) - Pero usar \textbf{modelos robustos}
(quantile regression, GLM gamma)

\paragraph{Outliers por Especie (Base
Balanceada)}\label{outliers-por-especie-base-balanceada}

\begin{verbatim}
Especie          Outliers_Precio  Outliers_Cantidad
-------------   ---------------  -----------------
ANCHOVETA       0                20 / 18 obs (111%)
JUREL           0                13 / 18 obs (72%)
SARDINA COMÚN   0                21 / 18 obs (117%)
\end{verbatim}

\textbf{Nota:} El número de outliers puede exceder N obs porque es suma
acumulada de outliers regionales.

\begin{center}\rule{0.5\linewidth}{0.5pt}\end{center}

\subsubsection{10. COMPOSICIÓN DE
FLOTAS}\label{composiciuxf3n-de-flotas}

\paragraph{Participación Industrial
Promedio}\label{participaciuxf3n-industrial-promedio}

\begin{verbatim}
Especie          % Industrial
--------------   ------------
ANCHOVETA        2.7%   (97.3% artesanal)
JUREL            93.8%  (6.2% artesanal)
SARDINA COMÚN    5.3%   (94.7% artesanal)
\end{verbatim}

\textbf{Conclusión:} - \textbf{Jurel = Flota industrial} -
\textbf{Sardina y Anchoveta = Flota artesanal}

\paragraph{Evolución Temporal (\%
Industrial)}\label{evoluciuxf3n-temporal-industrial}

\begin{verbatim}
Año   Anchoveta  Jurel  Sardina
----  ---------  -----  -------
2012  15.7%      98.8%  5.7%
2016  2.2%       99.2%  33.7%  ← Sardina tuvo pico industrial
2017  4.1%       99.6%  0.2%
2018  17.8%      97.2%  44.7%  ← Otro pico sardina
2020  0.0%       98.9%  0.0%   ← Artesanal puro
2022  1.2%       55.1%  0.1%   ← ¡Jurel cae a 55%!
2023  0.9%       98.8%  1.8%   ← Jurel recupera
2024  0.9%       97.8%  1.4%
\end{verbatim}

\textbf{Hallazgos:} 1. \textbf{2022 es ANÓMALO:} Jurel industrial cae a
55\% (¿regulación, veda, crisis?) 2. \textbf{Sardina fluctúa:} 0.2\% →
44.7\% → 0.1\% (muy variable) 3. \textbf{Anchoveta estable:} Siempre
\textless18\% industrial

\begin{center}\rule{0.5\linewidth}{0.5pt}\end{center}

\subsection{EVALUACIÓN METODOLÓGICA DEL
CÓDIGO}\label{evaluaciuxf3n-metodoluxf3gica-del-cuxf3digo}

\subsubsection{Fortalezas del Script}\label{fortalezas-del-script}

\begin{enumerate}
\def\labelenumi{\arabic{enumi}.}
\item
  \textbf{Estrategia dual bien pensada:}

  \begin{itemize}
  \tightlist
  \item
    Regional para heterogeneidad espacial
  \item
    Macrozonal para IAIDS
  \end{itemize}
\item
  \textbf{Ponderación correcta:}

\begin{Shaded}
\begin{Highlighting}[]
\NormalTok{PRECIO\_MACRO }\OtherTok{=} \FunctionTok{weighted.mean}\NormalTok{(PRECIO\_PONDERADO, }\AttributeTok{w =}\NormalTok{ Q\_MUESTRA\_PLANTAS)}
\end{Highlighting}
\end{Shaded}

  Pondera por volumen de muestra, no por número de regiones.
\item
  \textbf{Indicadores de calidad incluidos:}

  \begin{itemize}
  \tightlist
  \item
    \texttt{CV\_PRECIO\_REGIONAL}: Detecta segmentación
  \item
    \texttt{N\_OUTLIERS\_PRECIO/CANTIDAD}: Flaggea eventos extremos
  \item
    \texttt{N\_REGIONES\_PRECIO/DESEMB}: Documenta cobertura
  \end{itemize}
\item
  \textbf{Tres outputs diferenciados:}

  \begin{itemize}
  \tightlist
  \item
    Regional (análisis espacial)
  \item
    Macro completa (máxima data)
  \item
    Macro balanceada (IAIDS)
  \end{itemize}
\item
  \textbf{Validaciones exhaustivas:}

  \begin{itemize}
  \tightlist
  \item
    Diferencias regional vs macro
  \item
    Rangos razonables
  \item
    Casos especiales documentados
  \end{itemize}
\end{enumerate}

\subsubsection{Áreas de Mejora}\label{uxe1reas-de-mejora}

\paragraph{1. Imputación de Precios Faltantes
(Opcional)}\label{imputaciuxf3n-de-precios-faltantes-opcional}

\textbf{Problema:} Solo 18/107 meses tienen 3 especies.

\textbf{Solución posible:}

\begin{Shaded}
\begin{Highlighting}[]
\CommentTok{\# Imputar con media móvil regional}
\NormalTok{df\_precios\_imputado }\OtherTok{\textless{}{-}}\NormalTok{ df\_precios }\SpecialCharTok{\%\textgreater{}\%}
  \FunctionTok{group\_by}\NormalTok{(RG, NM\_RECURSO) }\SpecialCharTok{\%\textgreater{}\%}
  \FunctionTok{arrange}\NormalTok{(ANIO, MES) }\SpecialCharTok{\%\textgreater{}\%}
  \FunctionTok{mutate}\NormalTok{(}
    \AttributeTok{PRECIO\_IMPUTADO =}\NormalTok{ zoo}\SpecialCharTok{::}\FunctionTok{na.approx}\NormalTok{(PRECIO\_PONDERADO, }\AttributeTok{na.rm =} \ConstantTok{FALSE}\NormalTok{),}
    \AttributeTok{PRECIO\_ORIGINAL =} \SpecialCharTok{!}\FunctionTok{is.na}\NormalTok{(PRECIO\_PONDERADO)}
\NormalTok{  )}
\end{Highlighting}
\end{Shaded}

\textbf{Ventajas:} - Crea panel más completo (maybe 50-60 meses
balanceados)

\textbf{Desventajas:} - Introduce supuestos (interpolación lineal) -
Debes reportar sensibilidad con/sin imputación

\paragraph{2. Deflactar Precios}\label{deflactar-precios}

\textbf{Problema:} Serie 2012-2024 (12 años, inflación acumulada
\textasciitilde60\% en Chile).

\textbf{Solución:}

\begin{Shaded}
\begin{Highlighting}[]
\NormalTok{df\_macro }\OtherTok{\textless{}{-}}\NormalTok{ df\_macro }\SpecialCharTok{\%\textgreater{}\%}
  \FunctionTok{left\_join}\NormalTok{(ipc\_data, }\AttributeTok{by =} \StringTok{"ANIO"}\NormalTok{) }\SpecialCharTok{\%\textgreater{}\%}
  \FunctionTok{mutate}\NormalTok{(}
    \AttributeTok{PRECIO\_REAL =}\NormalTok{ PRECIO\_MACRO }\SpecialCharTok{*}\NormalTok{ (IPC\_2024 }\SpecialCharTok{/}\NormalTok{ IPC),}
    \AttributeTok{PRECIO\_NOMINAL =}\NormalTok{ PRECIO\_MACRO}
\NormalTok{  )}
\end{Highlighting}
\end{Shaded}

\paragraph{3. Agregar Precio
Internacional}\label{agregar-precio-internacional}

\textbf{Fuente:} IFFO (harina de pescado Perú super prime)

\begin{Shaded}
\begin{Highlighting}[]
\CommentTok{\# Merge con serie internacional}
\NormalTok{df\_final }\OtherTok{\textless{}{-}}\NormalTok{ df\_macro }\SpecialCharTok{\%\textgreater{}\%}
  \FunctionTok{left\_join}\NormalTok{(precio\_harina\_mundial, }\AttributeTok{by =} \FunctionTok{c}\NormalTok{(}\StringTok{"ANIO"}\NormalTok{, }\StringTok{"MES"}\NormalTok{))}
\end{Highlighting}
\end{Shaded}

\textbf{Importancia:} El precio ex-vessel está \textbf{fuertemente
correlacionado} con precio de harina.

\paragraph{4. Variables Derivadas}\label{variables-derivadas}

\begin{Shaded}
\begin{Highlighting}[]
\CommentTok{\# Rezagos}
\NormalTok{df\_final }\OtherTok{\textless{}{-}}\NormalTok{ df\_final }\SpecialCharTok{\%\textgreater{}\%}
  \FunctionTok{group\_by}\NormalTok{(NM\_RECURSO) }\SpecialCharTok{\%\textgreater{}\%}
  \FunctionTok{arrange}\NormalTok{(FECHA) }\SpecialCharTok{\%\textgreater{}\%}
  \FunctionTok{mutate}\NormalTok{(}
    \AttributeTok{P\_LAG1 =} \FunctionTok{lag}\NormalTok{(PRECIO\_MACRO, }\DecValTok{1}\NormalTok{),}
    \AttributeTok{Q\_LAG1 =} \FunctionTok{lag}\NormalTok{(Q\_MACRO, }\DecValTok{1}\NormalTok{),}
    
    \CommentTok{\# Promedio móvil}
    \AttributeTok{P\_MA3 =}\NormalTok{ (PRECIO\_MACRO }\SpecialCharTok{+} \FunctionTok{lag}\NormalTok{(PRECIO\_MACRO,}\DecValTok{1}\NormalTok{) }\SpecialCharTok{+} \FunctionTok{lag}\NormalTok{(PRECIO\_MACRO,}\DecValTok{2}\NormalTok{)) }\SpecialCharTok{/} \DecValTok{3}\NormalTok{,}
    
    \CommentTok{\# Variación mensual}
    \AttributeTok{P\_CRECIMIENTO =}\NormalTok{ (PRECIO\_MACRO }\SpecialCharTok{/} \FunctionTok{lag}\NormalTok{(PRECIO\_MACRO, }\DecValTok{1}\NormalTok{) }\SpecialCharTok{{-}} \DecValTok{1}\NormalTok{) }\SpecialCharTok{*} \DecValTok{100}
\NormalTok{  )}
\end{Highlighting}
\end{Shaded}

\begin{center}\rule{0.5\linewidth}{0.5pt}\end{center}

\subsection{PROBLEMAS IDENTIFICADOS Y
SOLUCIONES}\label{problemas-identificados-y-soluciones}

\subsubsection{Problema 1: Base Balanceada MUY PEQUEÑA (54
obs)}\label{problema-1-base-balanceada-muy-pequeuxf1a-54-obs}

\textbf{Impacto:} - IAIDS requiere estimar múltiples parámetros - Con
solo 54 obs, poder estadístico es bajo - Intervalos de confianza serán
amplios

\textbf{Soluciones:}

\textbf{Opción A: Imputar precios faltantes}

\begin{Shaded}
\begin{Highlighting}[]
\CommentTok{\# Pros: Sube a \textasciitilde{}150{-}180 obs}
\CommentTok{\# Contras: Introduce supuestos}
\CommentTok{\# Recomendación: Hacer y reportar robustez}
\end{Highlighting}
\end{Shaded}

\textbf{Opción B: Ampliar ventana temporal}

\begin{Shaded}
\begin{Highlighting}[]
\CommentTok{\# Incluir 2010{-}2011 si hay datos}
\CommentTok{\# Pros: Más observaciones}
\CommentTok{\# Contras: Puede haber cambios estructurales}
\end{Highlighting}
\end{Shaded}

\textbf{Opción C: Modelo parcial}

\begin{Shaded}
\begin{Highlighting}[]
\CommentTok{\# Estimar con pares de especies cuando solo 2 disponibles}
\CommentTok{\# Pros: Usa más data (107 meses)}
\CommentTok{\# Contras: Más complejo, resultados menos directos}
\end{Highlighting}
\end{Shaded}

\textbf{Recomendación:} - \textbf{Principal:} Usa base balanceada (54
obs) con resultados cautelosos - \textbf{Robustez:} Re-estima con base
imputada

\begin{center}\rule{0.5\linewidth}{0.5pt}\end{center}

\subsubsection{Problema 2: Correlaciones Precio-Cantidad
Positivas}\label{problema-2-correlaciones-precio-cantidad-positivas}

\textbf{Diagnóstico:} Variable omitida (precio internacional).

\textbf{Solución:}

\begin{Shaded}
\begin{Highlighting}[]
\CommentTok{\# Modelo con precio harina}
\FunctionTok{lm}\NormalTok{(}\FunctionTok{log}\NormalTok{(PRECIO\_MACRO) }\SpecialCharTok{\textasciitilde{}} \FunctionTok{log}\NormalTok{(Q\_MACRO) }\SpecialCharTok{+} \FunctionTok{log}\NormalTok{(PRECIO\_HARINA\_INTL) }\SpecialCharTok{+} 
                       \FunctionTok{factor}\NormalTok{(NM\_RECURSO) }\SpecialCharTok{+} \FunctionTok{factor}\NormalTok{(MES))}
\end{Highlighting}
\end{Shaded}

\textbf{Esperado:} - \texttt{log(Q\_MACRO)} se vuelve negativo
(correcto) - \texttt{log(PRECIO\_HARINA\_INTL)} captura tendencia común

\begin{center}\rule{0.5\linewidth}{0.5pt}\end{center}

\subsubsection{Problema 3: Diferencias Cantidad Regional vs
Macro}\label{problema-3-diferencias-cantidad-regional-vs-macro}

\textbf{Causa:} Base regional solo suma regiones CON precio.

\textbf{Solución:} Documenta claramente en tu tesis:

\begin{quote}
``La base macrozonal agrega desembarques de TODAS las regiones del
centro-sur, mientras que la base regional solo incluye regiones con
datos de precio disponibles. Por tanto, los desembarques macrozonas son
sistemáticamente mayores (promedio +19\%) que la suma regional reportada
en base\_regional.csv. Esta diferencia es esperada y no constituye un
error.''
\end{quote}

\begin{center}\rule{0.5\linewidth}{0.5pt}\end{center}

\subsection{CHECKLIST ANTES DE ESTIMAR
IAIDS}\label{checklist-antes-de-estimar-iaids}

\subsubsection{Datos}\label{datos}

\begin{itemize}
\tightlist
\item[$\boxtimes$]
  Base balanceada con 3 especies × N meses
\item[$\boxtimes$]
  Precios y cantidades positivos
\item[$\boxtimes$]
  Sin valores faltantes
\item[$\square$]
  Precios deflactados (PENDIENTE - recomendado)
\item[$\square$]
  Precio internacional incluido (PENDIENTE - crítico)
\end{itemize}

\subsubsection{Variables Derivadas}\label{variables-derivadas-1}

\begin{itemize}
\tightlist
\item[$\square$]
  Shares de gasto calculados
\item[$\square$]
  Índice de precios Stone
\item[$\square$]
  Variables de control (tendencia, estacionalidad)
\end{itemize}

\subsubsection{Validaciones}\label{validaciones}

\begin{itemize}
\tightlist
\item[$\boxtimes$]
  Consistencia agregación regional-macro
\item[$\boxtimes$]
  Rangos razonables de precios/cantidades
\item[$\boxtimes$]
  Outliers identificados
\item[$\square$]
  Multicolinealidad verificada
\item[$\square$]
  Estacionariedad testada
\end{itemize}

\begin{center}\rule{0.5\linewidth}{0.5pt}\end{center}

\subsection{📊 ESTADÍSTICAS FINALES: BASE MACROZONAL
BALANCEADA}\label{estaduxedsticas-finales-base-macrozonal-balanceada}

\subsubsection{Periodo y Cobertura}\label{periodo-y-cobertura}

\begin{verbatim}
Total observaciones: 54 (18 meses × 3 especies)
Período: 2012-2024 (13 años, pero solo 18 meses con datos completos)
Distribución temporal:
  - 2012-2018: 4 meses (22%)
  - 2020-2024: 14 meses (78%)  ← Sesgo reciente
\end{verbatim}

\subsubsection{Precios (\$/tonelada)}\label{precios-tonelada}

\begin{longtable}[]{@{}llllll@{}}
\toprule\noalign{}
Especie & Media & Mediana & Min & Max & CV \\
\midrule\noalign{}
\endhead
\bottomrule\noalign{}
\endlastfoot
Anchoveta & 151,781 & 149,826 & 62,695 & 215,570 & 24\% \\
Jurel & 219,157 & 217,696 & 59,000 & 300,000 & 35\% \\
Sardina & 150,271 & 150,636 & 63,621 & 215,570 & 23\% \\
\end{longtable}

\subsubsection{Cantidades
(toneladas/mes)}\label{cantidades-toneladasmes}

\begin{longtable}[]{@{}llllll@{}}
\toprule\noalign{}
Especie & Media & Mediana & Min & Max & CV \\
\midrule\noalign{}
\endhead
\bottomrule\noalign{}
\endlastfoot
Anchoveta & 24,682 & 15,957 & 1,274 & 118,707 & 152\% \\
Jurel & 70,006 & 54,458 & 23,188 & 360,130 & 140\% \\
Sardina & 55,606 & 33,728 & 15,462 & 358,354 & 158\% \\
\end{longtable}

\textbf{Nota:} Alta variabilidad en cantidades (CV \textgreater140\%) vs
baja en precios (CV \textless35\%).

\begin{center}\rule{0.5\linewidth}{0.5pt}\end{center}

\subsection{RECOMENDACIONES FINALES}\label{recomendaciones-finales}

\subsubsection{Para Estimación IAIDS}\label{para-estimaciuxf3n-iaids}

\begin{enumerate}
\def\labelenumi{\arabic{enumi}.}
\item
  \textbf{Usa base\_macrozonal\_balanceada.csv}

  \begin{itemize}
  \tightlist
  \item
    Es perfectamente balanceada (requisito IAIDS)
  \item
    Tiene 54 obs (suficiente pero justo)
  \end{itemize}
\item
  \textbf{Deflacta precios ANTES de estimar}

  \begin{itemize}
  \tightlist
  \item
    Fundamental para serie 2012-2024
  \item
    Usa IPC Chile o Índice de Precios al Productor
  \end{itemize}
\item
  \textbf{Agrega precio internacional harina}

  \begin{itemize}
  \tightlist
  \item
    Variable de control crítica
  \item
    Explica tendencia común en precios
  \end{itemize}
\item
  \textbf{Especificación recomendada:}

\begin{verbatim}
Share_i = α_i + Σ γ_ij log(P_j) + β_i log(Y/P*) + 
          δ_i PRECIO_HARINA + θ_i TENDENCIA + ε_i
\end{verbatim}
\item
  \textbf{Análisis de robustez:}

  \begin{itemize}
  \tightlist
  \item
    Con/sin outliers
  \item
    Con/sin controles temporales
  \item
    Ventana móvil (últimos 5 años vs serie completa)
  \end{itemize}
\end{enumerate}

\subsubsection{Para tu Tesis}\label{para-tu-tesis}

\begin{enumerate}
\def\labelenumi{\arabic{enumi}.}
\item
  \textbf{Documenta limitaciones:}

  \begin{itemize}
  \tightlist
  \item
    Solo 18 meses balanceados de 107 disponibles (16.8\%)
  \item
    Sesgo temporal hacia 2020-2024 (78\% de datos)
  \item
    Correlaciones precio-cantidad contraintuitivas
  \end{itemize}
\item
  \textbf{Explica decisiones metodológicas:}

  \begin{itemize}
  \tightlist
  \item
    Por qué macrozonal vs regional
  \item
    Por qué base balanceada (requisito técnico IAIDS)
  \item
    Cómo trataste outliers (flaggeados, no eliminados)
  \end{itemize}
\item
  \textbf{Reporta tabla comparativa:}

\begin{verbatim}
| Modelo | Base | N | R² | Elasticidades |
|--------|------|---|----|--------------| 
| ...
\end{verbatim}
\item
  \textbf{Análisis de sensibilidad:}

  \begin{itemize}
  \tightlist
  \item
    Tabla con resultados usando las 3 bases
  \item
    Muestra robustez (o falta de ella)
  \end{itemize}
\end{enumerate}

\begin{center}\rule{0.5\linewidth}{0.5pt}\end{center}

Tu trabajo de integración es \textbf{EXCELENTE}. El código está bien
estructurado, las validaciones son exhaustivas, y las decisiones
metodológicas son correctas.

\end{document}
